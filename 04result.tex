\chapter{解析結果}
本研究は,ゴルフスイングを数値化したデータに,ヒルベルト・ファン変換を適用させ,瞬時周波数領域で解析を行う.

%=================================================================================================================================================================================================================================%
\section{ゴルフスイングの数値化}
ゴルフスイングの数値化は,慣性式モーションキャプチャを使用して行う.
使用するモーションキャプチャは,PERCEPTION NEURON 2.0を使用する.
\begin{figure}
    \begin{center}
        \includegraphics[width=3.5cm]{./images/sensors.png}
        \caption{加速度センサがついている位置}
        \label{sensors}
    \end{center}
\end{figure}
PERCEPTION NEURON 2.0は,図\ref{sensors}のように17点の位置に加速度センサを装着する.
この加速度センサより,推定の位置座標と$x$,$y$,$z$軸方向の回転角度を時系列にキャプチャしrawデータに書き出す.

PERCEPTION NEURON 2.0より,ゴルフスイングの時系列データを採取したrawデータであるため,bvh(biovision hierarchy)ファイルに変換する必要がある.
\begin{figure}
    \begin{center}
        \includegraphics[width=5cm]{./images/Tpose.png}
        \caption{T-pose}
        \label{tpose}
    \end{center}
\end{figure}
bvhファイルとは,図\ref{tpose}のように指定された点の推定の位置座標や$x$,$y$,$z$軸方向の回転角を時系列に書き出したファイルである.
rawデータからbvhファイルの変換は,PERCEPTION NEURON 2.0専用ソフトであるAxis Neuronより行う.

図\ref{tpose}は,T-poseと呼ばれる,デフォルトポーズである.
このポーズより$x$,$y$,$z$軸方向を定義し,各関節球の$x$,$y$,$z$軸の回転角を0度とする.
bvhファイルに書き込まれている時系列データは,このポーズからの回転角を示し,ルートである腰部は$x$,$y$,$z$方向の位置座標も記録する.
よって,各関節球の3軸方向の回転角と腰部の3軸方向の位置座標を合計し,180チャンネルの時系列データを扱う.

%=================================================================================================================================================================================================================================%
\section{被験者情報}
被験者は,ゴルフ歴10年,平均スコア100のアベレージゴルファーである.
被験者にドライバーショットを行わせたところ,ストレート弾道に飛球したゴルフスイング,スライス弾道でヘッドアップ動作をしたゴルフスイング,スライス弾道で身体が開く動作をしたゴルフスイングの3種類採取することができ,各種類で6スイングずつの時系列データを採取することができた.
各時系列データごとにヒルベルト・ファン変換を適用させたところ,4から6個のIMFと残差に分解した.
本研究では,各IMF毎に瞬時周波数と瞬時振幅を求め,瞬時周波数$f$,瞬時振幅$A$は以下の式のように処理をした.

\begin{equation}
    f_{ave}(t) = \frac{1}{n} \sum_{k=1}^{n} f_{k}
\end{equation}

\begin{equation}
    \tilde{A}(t) = \sqrt{\sum_{k=1}^{n} A_{k}}
\end{equation}

本研究では,アベレージゴルファーのスライスの原因としてよく挙げられるヘッドアップ動作,身体が開く動作をしたゴルフスイングに注目し,ストレート弾道に飛球したゴルフスイングと比較して考察を行う.

%=================================================================================================================================================================================================================================%
\section{スペクトログラム解析}
%=================================================================================================================================================================================================================================%
\subsection{頸部,左膝モーションのIMF1}
\begin{figure}
    \begin{center}
        \begin{tabular}{c}
            \begin{minipage}{0.5\hsize}
                \begin{center}
                    \includegraphics[width=8cm]{./images/straight_data/neck/IMF1.png}
                    % \caption{ストレート弾道で頸部モーションIMF1}
                    (a)
                    \label{straight neck imf1}
                \end{center}
            \end{minipage}

            \begin{minipage}{0.5\hsize}
                \begin{center}
                    \includegraphics[width=8cm]{./images/straight_data/left_leg/IMF1.png}
                    % \caption{ストレート弾道で左膝モーションIMF1}
                    (b)
                    \label{straight left leg imf1}
                \end{center}
            \end{minipage}
        \end{tabular}
    \end{center}

    \begin{center}
        \begin{tabular}{c}
            \begin{minipage}{0.5\hsize}
                \begin{center}
                    \includegraphics[width=8cm]{./images/straight_data/neck/IMF1.png}
                    % \caption{スライス弾道で頸部モーションIMF1}
                    (c)
                    \label{slice neck imf1}
                \end{center}
            \end{minipage}

            \begin{minipage}{0.5\hsize}
                \begin{center}
                    \includegraphics[width=8cm]{./images/straight_data/left_leg/IMF1.png}
                    % \caption{スライス弾道で左膝モーションIMF1}
                    (d)
                    \label{slice left leg imf1}
                \end{center}
            \end{minipage}
        \end{tabular}
    \end{center}
    \caption{ストレート弾道,スライス弾道に飛球した頸部,左膝モーションIMF1のスペクトログラム.(a)はストレート弾道で頸部モーションIMF1.(b)はストレート弾道で左膝モーションIMF1.(c)はスライス弾道で頸部モーションIMF1.(d)はスライス弾道で左膝モーションIMF1.}
    \label{imf1}
\end{figure}

% 被験者のドライバーショットを数値化し,ストレート弾道,ヘッドアップ動作をしたスライス弾道,身体が開く動作をしたスライス弾道の3つに分類し,各分類毎ヒルベルト・ファン変換を適用させ,瞬時周波数と瞬時振幅を計算した.
図\ref{imf1}は,ストレート弾道,スライス弾道に飛球したゴルフスイングモーションにHHTを適用させ,各弾道の頸部モーション,左膝モーションのIMF1をスペクトログラムにしたものある.
図\ref{imf1}の縦軸は瞬時周波数,横軸は時間,カラーバーは瞬時振幅(度),赤線はインパクト,緑線はトップを示している.
IMF1のスペクトログラムでは,瞬時周波数がおよそ$0\rm{Hz}$から$30\rm{Hz}$帯,瞬時振幅はおよそ0.4度から0.6度で離散的に分布されていることが確認できる.
すなわち,高周波成分のスペクトログラムでは,ストレート弾道とスライス弾道となる原因を示すことが難しい.

%=================================================================================================================================================================================================================================%
\subsection{ヘッドアップ動作}
\begin{figure}
    \begin{center}
        \begin{tabular}{c}
            \begin{minipage}{0.5\hsize}
                \begin{center}
                    \includegraphics[width=8cm]{./images/straight_data/neck/IMF4.png}
                    (e)
                \end{center}
            \end{minipage}

            \begin{minipage}{0.5\hsize}
                \begin{center}
                    \includegraphics[width=8cm]{./images/headup_data/neck/IMF4.png}
                    (f)
                \end{center}
            \end{minipage}
        \end{tabular}
    \end{center}
    \caption{ストレート弾道,スライス弾道に飛球した頸部モーションIMF4のスペクトログラム.(e)はストレート弾道の頸部モーションIMF4.(f)はスライス弾道の頸部モーションIMF4.}
\end{figure}

%=================================================================================================================================================================================================================================%
\subsection{身体が開く動作}
\begin{figure}
    \begin{center}
        \begin{tabular}{c}
            \begin{minipage}{0.5\hsize}
                \begin{center}
                    \includegraphics[width=8cm]{./images/straight_data/left_up_leg/IMF4.png}
                    (g)
                \end{center}
            \end{minipage}

            \begin{minipage}{0.5\hsize}
                \begin{center}
                    \includegraphics[width=8cm]{./images/straight_data/left_leg/IMF4.png}
                    (h)
                \end{center}
            \end{minipage}
        \end{tabular}
    \end{center}

    \begin{center}
        \begin{tabular}{c}
            \begin{minipage}{0.5\hsize}
                \begin{center}
                    \includegraphics[width=8cm]{./images/opening_data/left_up_leg/IMF4.png}
                    (i)
                \end{center}
            \end{minipage}

            \begin{minipage}{0.5\hsize}
                \begin{center}
                    \includegraphics[width=8cm]{./images/opening_data/left_up_leg/IMF4.png}
                    (j)
                \end{center}
            \end{minipage}
        \end{tabular}
    \end{center}
    \caption{ストレート弾道,スライス弾道に飛球した左腿,左膝モーションIMF4のスペクトログラム.(g)はストレート弾道の左腿モーションIMF4.(h)はストレート弾道の左膝モーションIMF4.(i)はスライス弾道の左腿モーションIMF4.(h)はスライス弾道の左膝モーションIMF4.}
\end{figure}
