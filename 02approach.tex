\chapter{ヒルベルト-ファン変換}
%
ヒルベルト-ファン変換(HHT : Hilbert Huang Transform)とは,
信号$x(t)$ を,経験的モード分解(EMD : Empirical Mode Decomposition)より,有限の固有モード関数(IMF : Intrinsic Mode Function)$\sum_{k=1}^{n}{\rm{IMF}}_{k}$ と一つの残差に分解し,
分解した$\rm{IMF}$ にヒルベルト変換を適用させ,瞬時周波数$\omega(t)$ と瞬時振幅$A(t)$ を求める手法である.
%

瞬時周波数$\omega(t)$ と瞬時振幅$A(t)$の求め方は,解析信号$z(t) = z_{r}(t) + iz_{i}(t)$の実部$z_{r}(t)$ と虚部$z_{i}(t)$から以下の式のように求める.
\begin{equation}
    \label{inst freq}
    \omega(t) = \frac{d}{dt}\arctan\frac{z_{i}(t)}{z_{r}(t)}
\end{equation}

\begin{equation}
    \label{inst amp}
    A(t) = |z(t)|
\end{equation}

また,解析信号の虚部である$z_{i}(t)$ を求めるために,実部$z_{r}(t)$ を経験的モード分解より得られた$\sum_{k=1}^{n}{\rm{IMF}}_{k}$ とし,その実部$z_{r}(t)$にヒルベルト変換を適用させることで,虚部$z_{i}(t)$ を求めることができる.
以下は,$z_{r}(t) = \sum_{k=1}^{n}{\rm{IMF}}_{k}$ にヒルベルト変換を適用させた式である.
\begin{equation}
    z_{i}(t) = \frac{1}{\pi}PV \int_{-\infty}^{\infty} \frac{z_{r}(t)}{t - \tau} d \tau = \frac{1}{\pi t} * z_{r}(t)
\end{equation}

%

\section{経験的モード分解}

\section{多変量経験的モード分解}

