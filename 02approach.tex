\chapter{ヒルベルト・ファン変換}
%
ヒルベルト・ファン変換(HHT : Hilbert Huang Transform)とは,
信号$x(t)$ を,経験的モード分解(EMD : Empirical Mode Decomposition)より,有限の固有モード関数(IMF : Intrinsic Mode Function)$\sum_{k=1}^{n}{\rm{IMF}}_{k}$ と一つの残差に分解し,
分解した${\rm{IMF}}_{k}$ にヒルベルト変換を適用させ,瞬時周波数$\omega(t)$ と瞬時振幅$A(t)$ を求める手法である.
%

% 瞬時周波数$\omega(t)$ と瞬時振幅$A(t)$の求め方は,解析信号$z(t) = z_{r}(t) + iz_{i}(t)$の実部$z_{r}(t)$ と虚部$z_{i}(t)$から以下の式のように求める.
% \begin{equation}
%     \label{inst freq}
%     \omega(t) = \frac{d}{dt}\arctan\frac{z_{i}(t)}{z_{r}(t)}
% \end{equation}

% \begin{equation}
%     \label{inst amp}
%     A(t) = |z(t)|
% \end{equation}

% また,解析信号の虚部である$z_{i}(t)$ を求めるために,実部$z_{r}(t)$ を経験的モード分解より得られた$\sum_{k=1}^{n}{\rm{IMF}}_{k}$ とし,その実部$z_{r}(t)$にヒルベルト変換を適用させることで,虚部$z_{i}(t)$ を求めることができる.
% 以下は,$z_{r}(t) = \sum_{k=1}^{n}{\rm{IMF}}_{k}$ にヒルベルト変換を適用させた式である.$\rm{PV}$ は,コーシーの主値である.
% \begin{equation}
%     z_{i}(t) = \frac{1}{\pi}PV \int_{-\infty}^{\infty} \frac{z_{r}(t)}{t - \tau} d \tau = \frac{1}{\pi t} * z_{r}(t)
% \end{equation}
% 次節からは,HHTで使用されているEMDのアルゴリズム,多変量に拡張された多変量経験的モード分解について説明する.
%
%=================================================================================================================================================================================================================================%
\section{EMD}
%
EMDとは,信号$x(t)$ が有限の固有モード関数$\rm{IMF}_{k}$ と残差$r(t)$ で構成されていると仮定し,ヒューリスティックに分解する.
EMDの式を以下で示す.
\begin{equation}
    \label{emd}
    x(t) = \sum_{k=1}^{n} {\rm{IMF}}_{k} + r(t)
\end{equation}

${\rm{IMF}}_{k}$ は,以下の2つの条件を満たす.
\begin{itemize}
    \item 局所的極値の数とゼロ交差の数が0,または1であること.
    \item 局所的極値から構成された上側包絡線と下側包絡線の平均値が0であること.
\end{itemize}
%
EMDのアルゴリズムを以下に示す.
\begin{enumerate}
    \item 残差を計算.($r(t) = x(t)$ とする.)
    \begin{equation}
        r(t) = \sum_{k=1}^{n} {\rm{IMF}}_{k} - x(t)
    \end{equation}
    \item ${\rm{IMF}}_{\rm{old}}(t) = r(t)$ と初期化して,${\rm{IMF}}_{k}$を取り出す.
    \begin{enumerate}
        \item ${\rm{IMF}}_{\rm{old}}(t)$ の極大値を結ぶ包絡線$u(t)$ と,極小値を結ぶ包絡線$l(t)$ を三次スプライン補完で求め,$u(t)$ と$l(t)$ の平均を${\rm{IMF}}_{\rm{old}}(t)$ から引く.
        \begin{equation}
            {\rm{IMF}}_{\rm{new}}(t) = {\rm{IMF}}_{\rm{old}}(t) - \frac{u(t) - l(t)}{2}
        \end{equation}
        \item ${\rm{IMF}}_{\rm{new}}(t)$ が収束条件SD$(0.2 \leq SD \leq 0.3)$ を満たす場合,${\rm{IMF}}$ 集合に追加し,満たさない場合は(a),(b)を繰り返す.SDの収束条件は以下の式である.
        \begin{equation}
            SD = \sum_{t=1}^{n} \frac{({\rm{IMF}}_{\rm{old}}(t) - {\rm{IMF}}_{\rm{new}}(t) )^2}{{\rm{IMF}}_{\rm{new}}(t)^2 }
        \end{equation}
        
    \end{enumerate}
    \item $\sum_{k=1}^{n} {\rm{IMF}}_{k}$ が全て取り出されるまで,1,2を繰り返す.
\end{enumerate}

%
%=================================================================================================================================================================================================================================%
\section{多変量経験的モード分解}
%
一般に,多チャンネルに拡張された経験的モード分解として,多変量経験的モード分解(MEMD : Multivariate EMD)が提案されている.
本研究では,複数のセンサからヒトの動作を採取するため,MEMDを採用する.
%
%=================================================================================================================================================================================================================================%
\section{解析信号}
%
解析信号は,実信号が負の周波数成分を持たない信号として定義される.
以下に,解析信号の式を示す.%参考文献チェック
\begin{equation}
    \label{analyse signal}
    z(t) = z_{r}(t) + iz_{i}(t)
\end{equation}
本節では,解析信号の導出について説明する.

実信号$z_{r}(t) = x(t)$のスペクトル$X(\omega)$とし,$x(t)$と$X(\omega)$を以下の式で表現する.
\begin{equation}
    \label{x(t)}
    x(t) = \frac{1}{\sqrt{2\pi}} \int X(\omega) e^{i \omega t} d\omega
\end{equation}
%
\begin{equation}
    \label{X(t)}
    X(\omega) = \frac{1}{\sqrt{2\pi}} \int x(t) e^{-i\omega t} dt
\end{equation}
%
複素信号$z(t)$は,スペクトル$X(t)$のせいの周波数成分のみからなることから,次の式のように,$X(t)$の正の周波数部分のフーリエ逆変換として得られる.
\begin{equation}
    \label{iF(t)}
    z(t) = 2 \frac{1}{\sqrt{2\pi}} \int_{0}^{\infty} X(\omega) e^{i\omega t} d\omega
\end{equation}
%
\ref{iF(t)}式の係数$2$は,解析信号の実部を$x(t)$と等しくするためのものである.
よって,\ref{X(t)}式を用いると\ref{iF(t)}式は,
%
\begin{eqnarray}
    z(t) &=& 2 \frac{1}{2\pi} \int_{0}^{\infty} \int x(t^{\prime}) e^{-i\omega t^{\prime}} e^{i\omega t} dt^{\prime} d\omega \\
         &=& \frac{1}{\pi} \int_{0}^{\infty} \int x(t^{\prime}) e^{i\omega (t - t^{\prime})} dt^{\prime} d\omega \label{z(t)変換途中1}
\end{eqnarray}
%
% ここで,デルタ関数$\int_{0}^{\infty} e^{j\omega x} d\omega = \pi \delta (x) + \frac{j}{x}$より,
ここで,ステップ関数$u(\omega)$のフーリエ変換式は以下の式である.
\begin{equation}
    \label{ft step function}
    \int_{0}^{\infty} u(\omega) e^{i \omega x} = \pi \delta (x) + \frac{i}{x}
\end{equation}
%
\ref{ft step function}式を用いると,\ref{z(t)変換途中1}式は,
\begin{equation}
    z(t) = \frac{1}{\pi} \int x(t^{\prime}) \biggl[ \pi \delta (t - t^{\prime}) + \frac{i}{t - t^{\prime}} \biggr] dt^{\prime}
\end{equation}
%
となるため,結果以下の式のように解析信号が導出される.
\begin{equation}
    \label{解析信号オリジナル}
    z(t) = x(t) + \frac{i}{\pi} \int \frac{x(t^{\prime})}{t - t^{\prime}} dt^{\prime}
\end{equation}
%
ここで,$x(t) = z_{r}(t)$に戻し,\ref{解析信号オリジナル}式の第2項を$z_{i}(t)$とおくと,$z_{i}(t)$は以下の式のように表現することができる.
\begin{equation}
    \label{ヒルベルト変換}
    z_{i}(t) = H[z_{r}(t)] = \frac{1}{\pi} \int \frac{x(t^{\prime})}{t - t^{\prime}} dt^{\prime}
\end{equation}
%
%=================================================================================================================================================================================================================================%
\section{ヒルベルト変換}
%参考文献チェック
前節では,解析信号の導出について説明したが,その虚部$z_{i}(t)$は,\ref{ヒルベルト変換}式よりヒルベルト変換で求められることを示した.
ヒルベルト変換とは,与えられた関数の周波数成分の位相を$-\frac{\pi}{2}$ずらす操作である.
以下は,信号$x(t)$のスペクトルを$X(t)$とした時のヒルベルト変換式の導出を示す.

実信号$x(t) =\cos (\omega t)$と直交する虚部は,
\begin{equation}
    \label{cosから90度位相を遅らせた式}
    \cos (\omega t - \frac{\pi}{2}) =\sin (\omega t)
\end{equation}
である.
%
よって,実部$\cos (\omega t + \phi)$,虚部を\ref{cosから90度位相を遅らせた式}式より$\sin (\omega t + \phi)$とし,複素指数関数を立式すると以下の式のようになる.
\begin{equation}
    \label{複素指数関数}
    e^{i(\omega t + \phi)} = \cos (\omega t + \phi) + i \sin(\omega t + \phi)
\end{equation}
\ref{複素指数関数}式に,虚数単位$-i$を1回かけると
\begin{equation}
    \label{-i1回目}
    -i e^{i(\omega t + \phi)} = -i \cos (\omega t + \phi) + \sin(\omega t + \phi)
\end{equation}
となる.
この時の実部は$\sin(\omega t + \phi)$となり,\ref{複素指数関数}式の実部より$\frac{\pi}{2}$遅れていることが確認できる.
%
さらに,\ref{-i1回目}式に$-i$をかけると,
\begin{equation}
    -e^{i(\omega t + \phi)} = -\cos(\omega t + \phi) - i \sin(\omega t + \phi)
\end{equation}
となり,実部に注目すると\ref{-i1回目}式の実部より,位相が$\frac{\pi}{2}$遅れていることが確認できる.
%
\begin{equation}
    -\cos(\omega t + \phi) = \sin(\omega t + \phi - \frac{\pi}{2})
\end{equation}
%
すなわち,虚数単位$i$をかけることで,$\frac{\pi}{2}$位相を遅らせることができる.
しかしながら,これは連続された関数のみに適応が可能であるため,離散的関数の位相を$\frac{\pi}{2}$遅らせるには,$-i {\rm{sgn}}(f)$をかける必要がある.
%

実信号$x(t)$のスペクトルを$X(t)$とする時,全ての周波数成分の位相を$\frac{\pi}{2}$遅らせるには,スペクトル$X(t)$に$-i {\rm{sgn}} (f)$をかけ,その結果をフーリエ逆変換する必要がある.
まず,$X(f)$に$-i{\rm{sgn}}(f)$を掛けた直行スペクトルを$X_{\perp}(t)$とすると,以下の式のように表現できる.
\begin{equation}
    \label{X(t)の直行スペクトル}
    X_{\perp}(f) = -i {\rm{sgn}} (f) X (f)
\end{equation}
%
次に,\ref{X(t)の直行スペクトル}式より,$X_{\perp} (f)$のフーリエ逆変換を行うことで,$x(t)$の直行スペクトル$x_{\perp}(t)$を得ることができる.
以下は,$x_{\perp}(t)$の式である.
\begin{equation}
    \label{x(t)垂直}
    x_{\perp} (t) = -i \int^{+\infty}_{-\infty} {\rm{sgn}} (f) X(f) e^{2 i \pi f t} df
\end{equation}
%
ここで,$-i {\rm{sgn}} (f)$のフーリエ逆変換は以下の式である.
\begin{equation}
    \label{シグナル関数のフーリエ変換}
    F^{-1} \lbrace -i {\rm{sgn}} (f) \rbrace = \frac{1}{\pi t}
\end{equation}
%
そして,$X(f)$のフーリエ逆変換は$x(t)$である.
よって,\ref{x(t)垂直}式は,以下の式のように書くことができる.
\begin{equation}
    \label{ヒルベルト変換}
    x {\perp} (t) = \frac{1}{\pi} \int^{+\infty}_{-\infty} \frac{1}{t - \tau} x(\tau) d\tau
\end{equation}
%
一般に\ref{ヒルベルト変換}式は,ヒルベルト変換の公式と言われている.

%