\chapter{ヒルベルト・ファン変換}
%
ヒルベルト・ファン変換(HHT : Hilbert Huang Transform)とは,
信号$x(t)$ を,経験的モード分解(EMD : Empirical Mode Decomposition)より,有限の固有モード関数(IMF : Intrinsic Mode Function)$\sum_{k=1}^{n}{\rm{IMF}}_{k}$ と一つの残差に分解し,
分解した${\rm{IMF}}_{k}$ にヒルベルト変換を適用させ,瞬時周波数$\omega(t)$ と瞬時振幅$A(t)$ を求める手法である.
%

瞬時周波数$\omega(t)$ と瞬時振幅$A(t)$の求め方は,解析信号$z(t) = z_{r}(t) + iz_{i}(t)$の実部$z_{r}(t)$ と虚部$z_{i}(t)$から以下の式のように求める.
\begin{equation}
    \label{inst freq}
    \omega(t) = \frac{d}{dt}\arctan\frac{z_{i}(t)}{z_{r}(t)}
\end{equation}

\begin{equation}
    \label{inst amp}
    A(t) = |z(t)|
\end{equation}

また,解析信号の虚部である$z_{i}(t)$ を求めるために,実部$z_{r}(t)$ を経験的モード分解より得られた$\sum_{k=1}^{n}{\rm{IMF}}_{k}$ とし,その実部$z_{r}(t)$にヒルベルト変換を適用させることで,虚部$z_{i}(t)$ を求めることができる.
以下は,$z_{r}(t) = \sum_{k=1}^{n}{\rm{IMF}}_{k}$ にヒルベルト変換を適用させた式である.$\rm{PV}$ は,コーシーの主値である.
\begin{equation}
    z_{i}(t) = \frac{1}{\pi}PV \int_{-\infty}^{\infty} \frac{z_{r}(t)}{t - \tau} d \tau = \frac{1}{\pi t} * z_{r}(t)
\end{equation}
次節からは,HHTで使用されているEMDのアルゴリズム,多変量に拡張された多変量経験的モード分解について説明する.
%

\section{EMD}
%
EMDとは,信号$x(t)$ が有限の固有モード関数$\rm{IMF}_{k}$ と残差$r(t)$ で構成されていると仮定し,ヒューリスティックに分解する.
EMDの式を以下で示す.
\begin{equation}
    \label{emd}
    x(t) = \sum_{k=1}^{n} {\rm{IMF}}_{k} + r(t)
\end{equation}

${\rm{IMF}}_{k}$ は,以下の2つの条件を満たす.
\begin{itemize}
    \item 局所的極値の数とゼロ交差の数が0,または1であること.
    \item 局所的極値から構成された上側包絡線と下側包絡線の平均値が0であること.
\end{itemize}
%
EMDのアルゴリズムを以下に示す.
\begin{enumerate}
    \item 残差を計算.($r(t) = x(t)$ とする.)
    \begin{equation}
        r(t) = \sum_{k=1}^{n} {\rm{IMF}}_{k} - x(t)
    \end{equation}
    \item ${\rm{IMF}}_{\rm{old}}(t) = r(t)$ と初期化して,${\rm{IMF}}_{k}$を取り出す.
    \begin{enumerate}
        \item ${\rm{IMF}}_{\rm{old}}(t)$ の極大値を結ぶ包絡線$u(t)$ と,極小値を結ぶ包絡線$l(t)$ を三次スプライン補完で求め,$u(t)$ と$l(t)$ の平均を${\rm{IMF}}_{\rm{old}}(t)$ から引く.
        \begin{equation}
            {\rm{IMF}}_{\rm{new}}(t) = {\rm{IMF}}_{\rm{old}}(t) - \frac{u(t) - l(t)}{2}
        \end{equation}
        \item ${\rm{IMF}}_{\rm{new}}(t)$ が収束条件SD$(0.2 \leq SD \leq 0.3)$ を満たす場合,${\rm{IMF}}$ 集合に追加し,満たさない場合は(a),(b)を繰り返す.SDの収束条件は以下の式である.
        \begin{equation}
            SD = \sum_{t=1}^{n} \frac{({\rm{IMF}}_{\rm{old}}(t) - {\rm{IMF}}_{\rm{new}}(t) )^2}{{\rm{IMF}}_{\rm{new}}(t)^2 }
        \end{equation}
        
    \end{enumerate}
    \item $\sum_{k=1}^{n} {\rm{IMF}}_{k}$ が全て取り出されるまで,1,2を繰り返す.
\end{enumerate}

%

\section{多変量経験的モード分解}
%
一般に,多チャンネルに拡張された経験的モード分解として,多変量経験的モード分解(MEMD : Multivariate EMD)が提案されている.
本研究では,複数のセンサからヒトの動作を採取するため,MEMDを採用する.
%
