\chapter{ヒルベルト-ファン変換}
%
ヒルベルト-ファン変換(HHT : Hilbert Huang Transform)とは,
信号$x(t)$ を,経験的モード分解(EMD : Empirical Mode Decomposition)より,有限の固有モード関数(IMF : Intrinsic Mode Function)$\sum_{k=1}^{n}{\rm{IMF}}_{k}$ と一つの残差に分解し,
分解した$\rm{IMF}$ にヒルベルト変換を適用させ,瞬時周波数$\omega(t)$ と瞬時振幅$A(t)$ を求める手法である.
%
瞬時周波数$\omega(t)$ と瞬時振幅$A(t)$ は,解析信号の実部と虚部から求めることができる.

解析信号は,以下の式で定義されている.
\begin{equation}
    \label{analitic signal}
    z(t) = z_{r}(t) + iz_{i}(t)
\end{equation}
ここで\ref{analitic signal}式より,実部$z_{r}$ と虚部$z_{i}$から,瞬時周波数と瞬時振幅を以下の式のように求める.
\begin{equation}
    A(t) = \sqrt{z_{r}{^{2}} + z_{i}{^{2}}}
    \omega(t) = 
\end{equation}

%

\section{経験的モード分解}

\section{多変量経験的モード分解}

\section{ヒルベルト変換}
